%%% PREAMBLE %%%

%% FORMATAÇÃO BÁSICA
\usepackage[english,brazilian]{babel} % Suporte para inglês e português (hifenação e caracteres especiais)
\usepackage[utf8]{inputenc} % Codificação do arquivo
\usepackage{cmap} % Mapear caracteres especiais no PDF
\usepackage[T1]{fontenc} % Codificação da fonte
\usepackage{times} % Fonte tradicional Serif
\usepackage{amsmath,amssymb,amsfonts,textcomp} % Funções e símbolos
\usepackage{enumerate} % Permite enumerar com carácteres vários
%% NEW COMMANDS
\newcommand{\shellcmd}[1]{\texttt{\$ #1}}
\newcommand{\userprompt}[1]{\texttt{alan@turing:$\sim$\$ #1}}
\newcommand{\rootprompt}[1]{\texttt{root@turing:$\sim$\# #1}}
\newcommand{\poyprompt}[1]{\texttt{poy> #1}}
\newcommand{\ie}{\textit{i.e.},}
\newcommand{\eg}{\textit{e.g.},}
%% LAYOUT
\usepackage{listings}
\usepackage{setspace} % Definir espaçamento
\setlength{\parskip}{0cm} % Espaçamento do texto para o frame
\usepackage{anysize} % Habilita edição de margens em uma página específica
\marginsize{1.6cm}{1.6cm}{1cm}{1cm}
\usepackage[portuguese]{minitoc}
% Formatação das headings de seções e subseções
\makeatletter
\renewcommand{\section}{\@startsection
{section} % the name
{1} % the level
{0mm} % the indent
{-\baselineskip} % the before skip
{0cm} % the after skip
{\normalfont\Small}} % the style
\makeatother
% Redefinindo heading \subsection
\makeatletter
\renewcommand{\subsection}{\@startsection{subsection}{1}{0mm}

{-\baselineskip}%
{0.1cm}{\normalfont\normalsize\scshape}}%
\makeatother
%% code blocks
\usepackage{minted}
\usepackage[most]{tcolorbox}
\definecolor{moonstoneblue}{rgb}{0.45, 0.66, 0.76}

%% ELEMENTOS GRÁFICOS
% Color text blocks
\usepackage{tcolorbox}
\tcbuselibrary{breakable}
\newenvironment{blackBlock}[1]{%
    \tcolorbox[%
    noparskip,breakable,
    colback=white,%
    colframe=gray,%
    title=#1]}%
    {\endtcolorbox}
\usepackage{graphicx} % Para incluir figuras (pacote extendido)
\usepackage{epstopdf} % Para incluir figuras .eps
\usepackage{wrapfig} % Para fazer o texto fluir através ou ao lado das figuras
%%%OBSERVACAO%%%
% Isso fará que a figura seja convertida de .eps para .pdf e então inserida do documento.
% Ainda assim talvez falte mais um detalhe. E você poderá continuar recebendo a mesma mensagem. O último passo é acessar o arquivo /etc/texmf/texmf.cnf
% Sugiro que use o nano:
% $ nano /etc/texmf/texmf.cnf 
% Procure a linha começando com 'shell_escape' e troque o valor de 'f' para 't'. Não mude mais nada, só isso. 
%%%%%%%%%%%%%
\usepackage{color, colortbl} % Suporte a cores
\usepackage[font=small,labelfont=bf]{caption} % Customizar as legendas de figuras e tabelas
\usepackage{floatrow} % Legendas do tamanho do objeto
\usepackage{multicol} % Criar ambientes com 2 ou mais colunas

%% TABELAS
\usepackage{array} % Elementos extras para formatação de tabelas
\usepackage{booktabs} % Tabelas com qualidade de publicação
\usepackage{longtable} % Para criar tabelas maiores que uma página
% \usepackage{lscape} % Adicionar tabelas e figuras como landscape
\usepackage{pdflscape} % Possibilita landscape no PDF
\usepackage{afterpage} % Permite que o texto flua através de uma página específica
\usepackage{changepage} % Permitir alterações no tamanho de páginas dentro do documento
\usepackage[table]{xcolor} % Permitir colora\cc\~ao de c\'elulas, http://ctan.org/pkg/xcolor

%% NOTAS DE RODAPÉ
\usepackage{footnote} % Lidar com notas de rodapé em diversas situações
\makesavenoteenv{tabular} % Notas criadas nas tabelas ficam no fim das tabelas

%% LINKS DINÂMICOS
\usepackage{hyperref} % Suporte para hipertexto, links para referências e figuras
\setcounter{secnumdepth}{6}
\setcounter{tocdepth}{6}
% Abaixo, configurações dos links e metatags do PDF a ser gerado
\hypersetup{colorlinks=true, linkcolor=blue, citecolor=blue, filecolor=blue, urlcolor=[rgb]{0,0.5,0.5},
            pdfauthor={Marques, Fernando Portella de Luna},
            pdftitle={TitleEnglish},
            pdfsubject={Subject},
            pdfkeywords={Keywords},
            pdfproducer={LaTeX},
            pdfcreator={PDFLaTeX}}
\usepackage{lastpage} % Conta o número de páginas

%% TABLE OF CONTENTS
\usepackage[nottoc,notlof,notlot]{tocbibind} % Não inclui lista de tabelas e figuras no índice

%% PONTUAÇÃO E UNIDADES
\usepackage{icomma} % Posicionar inteligentemente a vírgula como separador decimal
\usepackage[tight]{units} % Formatar as unidades com as distâncias corretas

%% CABEÇALHO E RODAPÉ
\usepackage{fancyhdr} % Controlar os cabeçalhos e rodapés
\pagestyle{fancy} % Usar os estilos do pacote fancyhdr
\fancypagestyle{plain}{\fancyhf{}}
\fancyhead{} % Limpar os campos do cabeçalho atual
%\fancyhead[RO]{\nouppercase{\small{Relatório Parcial Fapesp No. 2011/18947-9}}} % Cabeçalho: nome da seção do lado direito [R] em páginas ímpares [O]
\fancyhead[LE]{\nouppercase{\small{Marques, F.P.L.}}} % Cabeçalho: nome do autor do lado esquerdo [L] em páginas pares [E]
\renewcommand{\headrulewidth}{0pt} % Omitir linha de separação entre cabeçalho e conteúdo
\headheight 12pt % Altura do cabeçalho
\fancyfoot{} % Limpar os campos do rodapé
\fancyfoot[RO,LE]{\thepage} % Rodapé: número da página do lado direito [R] nas páginas ímpares [O] e do lado esquerdo [L] nas páginas pares [E]
\renewcommand{\footrulewidth}{0pt} % Omitir linha de separação entre rodapé e conteúdo

%% OUTROS COMANDOS CUSTOMIZADOS
\newenvironment{myindentpar}[1] % Ident full paragraph
 {\begin{list}{}
         {\setlength{\leftmargin}{#1}}
         \item[]
 }
 {\end{list}}

%% Pacotes adicionais
\usepackage{enumerate} % Permite editar a forme de enumerar itens

%% REDEFININDO OUTROS PACOTES E ESTILOS
% Redefinindo heading \section
\makeatletter
\renewcommand{\section}{\@startsection
{section} % the name
{1} % the level
{0mm} % the indent
{-\baselineskip} % the before skip
{0.1cm} % the after skip
{\normalfont\Small\bf}} % the style
\makeatother
% Redefinindo heading \subsection
\makeatletter
\renewcommand{\subsection}{\@startsection{subsection}{1}{0mm}
{-\baselineskip}%
{0.2cm}{\normalfont\normalsize\scshape}}%
\makeatother

% A3 page environment
\newenvironment{hugepage}
  % Text to replace \begin{hugepage}
  {\newpage \pagestyle{plain} \cleardoublepage
  % PDF output page --> A3
  \setlength{\pdfpagewidth}{11.7in}
  \setlength{\pdfpageheight}{16.5in}
  % \changepage{textheight}{textwidth}{evensidemargin}{oddsidemargin}{columnsep}{topmargin}{headheight}{headsep}{footskip}
  \changepage{146mm}{100mm}{5mm}{5mm}{}{0.1mm}{}{}{0.1mm}}
  % Text to replace \end{hugepage}
  {\cleardoublepage
  \pagestyle{fancy}}


%% IMPEDIMENTO DE HÍFENS
\hyphenation{% Liste abaixo as palavras que não devem ser divididas por hífens
Acanthobothium
ACANTHOBOTHRIE
Acanthobothrium
Acanthotaenia
aiereba
altavela
amazonensis
angelae
asotus
atrus
batis
Batoidea
bifurcatum
bifurcatus
boesemani
Bothriocephalus
Bothriuocephali
Bothrops
bovinus
Brachaelurus
brevissime
californica
Calophysus
Carcharhiniformes
cartagenensis
castexi
cemiculus
Chondrichthyes
clavata
corollata
corollatum
corollatus
coronatum
coronatus
COURONNÉ
Dasyatidae
Dasyatis
Dipturus
Echinocephalus
eglanteria
entemedor
etini
Eucestoda
Eutetrarhynchus
exanthematicus
fai
falkneri
fitzgeraldae
fluviatilis
Gangesia
gerrardi
glanis
gnomus
grabata
grandiceps
guttata
Gymnura
Gymnuridae
Heelandicum
jaenneae
japonicus
jarara
jararaca
kuhlii
laevis
lasti
lenha
leoparda
lobistoma
logus
macracanthum
macropterus
maculatus
margaritella
margieae
masnihae
mattaylori
megacephalum
mexicanus
microcephalum
miraletus
monopterygius
motoro
Myliobatidae
Myliobatiformes
Myliobatis
Narcine
Nematoda
Neotrygon
nipponensis
Nomimoscolex
oceanharvestae
onchobothrii
Onchobothriidae
Ophiotaenia
orbignyi
Orectolobiformes
Orectolobus
Paragaleus
parasiluri
Paratrygon
parviuncinatum
parvum
Pastinachus
pectoralis
Perca
percae
peruviense
Phractocephalus
Phractocephalus
Phyllobothiidae
piracatinga
pirarara
planiceps
Plesiotrygon
polylepis
popi
Potamotrygon
Potamotrygonidae
Potamotrygonocestus
Potomotrygon
Pristiophoriformes
Pristis
Proteocephalidea
Proteocephalus
Pteromylaeus
quinonesi
raiae
raiaebatis
raja
Raja
rajabatis
rajaebatis
Rajiformes
ramiroi
regoi
Rhinebothrium
Rhinebothroides
Rhiniformes
Rhinobatiformes
Rhinobatos
rhinobatos
Rhinobatos
Rhinobatos
Rhynchobatus
rodmani
romanowi
Rudolphiella
sabina
saliki
santarosaliense
say
schmardae
schoenleinii
schroederi
Scyliorhynus
septentrionale
siluri
Silurotaenia
Silurus
Sorubimichthys
Squatiniformes
squireorum
stellaris
Taenia
Taeniura
terezae
Tetraphyllidea
tetrastomus
Torpediniformes
tripartium
uarnacoides
Uncibilocularis
undulata
urarnak
Urobatis
Urotrygonidae
Varanus
velezi
waddi
walga
woodsholei
yepezi
zainili
Zanobatus
zimmeri
Zygobothriumliotrygon
hemioliopterus
herronorum
Heterodontiformes
Heterodontus
Himantura
histrix
holorhini
hypermekkolpos
Hypnos
Hypomesus
icelandicum
jaenneae
japonicus
jarara
jararaca
kuhlii
laevis
lasti
lenha
leoparda
lobistoma
logus
macracanthum
macropterus
maculatus
margaritella
margieae
masnihae
mattaylori
megacephalum
mexicanus
microcephalum
miraletus
monopterygius
motoro
Myliobatidae
Myliobatiformes
Myliobatis
Narcine
Nematoda
Neotrygon
nipponensis
Nomimoscolex
oceanharvestae
onchobothrii
Onchobothriidae
Ophiotaenia
orbignyi
Orectolobiformes
Orectolobus
Paragaleus
parasiluri
Paratrygon
parviuncinatum
parvum
Pastinachus
pectoralis
Perca
percae
peruviense
Phractocephalus
Phractocephalus
Phyllobothiidae
piracatinga
pirarara
planiceps
Plesiotrygon
polylepis
popi
Potamotrygon
Potamotrygonidae
Potamotrygonocestus
Potomotrygon
Pristiophoriformes
Pristis
Proteocephalidea
Proteocephalus
Pteromylaeus
quinonesi
raiae
raiaebatis
raja
Raja
rajabatis
rajaebatis
Rajiformes
ramiroi
regoi
Rhinebothrium
Rhinebothroides
Rhiniformes
Rhinobatiformes
Rhinobatos
rhinobatos
Rhinobatos
Rhinobatos
Rhynchobatus
rodmani
romanowi
Rudolphiella
sabina
saliki
santarosaliense
say
schmardae
schoenleinii
schroederi
Scyliorhynus
septentrionale
siluri
Silurotaenia
Silurus
Sorubimichthys
Squatiniformes
squireorum
stellaris
Taenia
Taeniura
terezae
Tetraphyllidea
tetrastomus
Torpediniformes
tripartium
uarnacoides
Uncibilocularis
undulata
urarnak
Urobatis
Urotrygonidae
Varanus
velezi
waddi
walga
woodsholei
yepezi
zainili
Zanobatus
zimmeri
Zygobothrium
}
